\documentclass[11pt,a4paper]{article}

\usepackage{placeins}
\usepackage{graphicx}
\usepackage{xcolor}
\usepackage{float}
\usepackage{afterpage}
\usepackage{amssymb,amsmath,bm}
\usepackage{multirow,booktabs,ulem}
\usepackage{cite}
\usepackage{jheppub}
\usepackage{hyperref}


\newcommand*{\tmp}[4]{\ensuremath{%
    {#4%
    \ifx\empty#3\empty\ifx\empty#1\empty\else^{#1}\fi\else^{#1(#3)}\fi%
    \ifx\empty#2\empty\else_{#2}\fi}%
}}
\newcommand*{\ccc}[4][]{\tmp{#2}{#3}{#4}{#1{c}}}
\newcommand{\sss}{\scriptscriptstyle}
\newcommand{\OO}{\ensuremath{\mathcal{O}}}
\newcommand{\Op}[1]{\OO_{\sss #1}}
\newcommand{\pdp}{\ensuremath{\varphi^\dagger\varphi}}
\def\lra#1{\overset{\text{\scriptsize$\leftrightarrow$}}{#1}}

\title{Collider signatures for ALPs and links with PDFs}

\begin{document}

\maketitle

The objective of this project is to fill some gaps in the pheno investigation of ALPs signatures at colliders.

\section{Collider stable ALPs and coupling to tops}

Start from ~\cite{Brivio:2017ije}.

\section{UnstableALPs and decay in top pairs}
Start from ~\cite{Bonilla:2021ufe}

\section{Coupling of ALPs to photons and gluons}
Starting from ~\cite{Mimasu:2014nea} and \cite{Gavela:2019cmq}, how
are the constraints modified if the ALP was part of the proton?

\bibliographystyle{utphys}
\bibliography{references}

\end{document}
